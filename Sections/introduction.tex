Diabetic Retinopathy is primarily caused by elevated levels of blood sugar over consistent, prolonged periods of time. Notably, elevated blood sugar levels increase the probability that the vasculature of the eye swells and/or leaks, leading to vision impairment [1]. When the vasculature of the retina swells or leaks, blood flow to distally positioned sections of the retina decreases, leading to long term retina damage and possibly the growth of new vascular networks. Both changes to the vasculature of the eye can severely impair normal vision, though the symptoms of Diabetic Retinopathy can be treated through a combination of early detection of the disease, lifestyle changes and medication. The most common and accurate method of diagnosing Diabetic Retinopathy is to image the retina using an optical coherence tomography (OCT) based system, though a more commonly used imaging modality is a FUNDUS based imaging system [1].

Given the dataset and background, the goal of this project is to compare the time needed to train a model, specifically a CNN classifier, over the dataset on both serial and parallel hardware. The serial hardware utilized in this project consists of a single CPU on Northeastern University's Discovery Compute Cluster, and the specialized hardware for parallel computations consists of GPUs (ranging from an NVIDIA T4 to an NVIDIA P100). The goal of this project is also to propose a CNN architecture that can predict the severity of Diabetic Retinopathy when trained on a set of FUNDUS images, with predictions exceeding the probability of assigning labels at random based on the class distributions of the data.
